\documentclass[sigconf]{acmart}

\usepackage{booktabs} % For formal tables

% Copyright
\setcopyright{rightsretained}

\begin{document}
\begin{titlepage}
\begin{center}

\textsc{\Large   Your Title }

\bigskip

\textsc{\large
submitted in partial fulfillment for the degree of master of science\\
%
\bigskip
Tom Akkermans\\
%
11323671\\
%
\bigskip
master information studies\\
%
data science \\
%
faculty of science\\
%
university of amsterdam\\
%
\bigskip
%Your Date of defence in the format YYYY-MM-DD
2018-08-DD
}

\end{center}
 

\vfill

% In case of an internal project, remove External Supervisor or if you had two internal supervisors, change the header into 
%  & First Supervisor & Second Supervisor  \\
\begin{center}
\begin{tabular}{|l||ll|}
\hline
 & \textbf{First Supervisor} & \textbf{Second Supervisor}  \\   
 \hline
\textbf{Title, Name} & Dr Maarten Marx&  \\
\textbf{Affiliation} &UvA, FNWI, IvI & \\ 
\textbf{Email} & maartenmarx@uva.nl& . \\
\hline
\end{tabular}
\end{center}



%% If you have a third supervisor use this table instead
%\begin{center}
%\begin{tabular}{|l||lll|}
%\hline
% & \textbf{External   Supervisor} & \textbf{External   Supervisor} & \textbf{3$^{\mathrm{rd}}$ supervisor} \\
% \hline
%\textbf{Title, Name} & Dr Maarten Marx& & \\
%\textbf{Affiliation} &UvA, FNWI, IvI & & \\ 
%\textbf{Email} & maartenmarx@uva.nl& &  .\\
%\hline
%\end{tabular}
%\end{center}

\bigskip

% logos
\begin{center}
\mbox{\includegraphics[width=.2\paperwidth]{logo-uva.png} 
\includegraphics[width=.2\paperwidth]{ads.png}
%\includegraphics[width=.2\paperwidth]{TitlePages/logos/ads.png} % replace by the logo of your internship company or remove
}
\end{center}
\end{titlepage}

%
%\newpage
%
%\end{document}

\pagebreak
\title{Data Driven Supervision in Insolvency Procedures}

\author{Tom Akkermans}
\affiliation{%
  \institution{University of Amsterdam}
}
\email{tom.akkermans@student.uva.nl}


\begin{abstract}
After a company files for bankruptcy at the court, a court committee will appoint a curator to settle the bankruptcy. The curator dissolves the company by monetizing the company's estate and redeeming its creditors. A court’s judge supervises the curator. Both these processes: curator appointment and supervision, have been topic of research[1-2], have raised concerns in the media[13-21] and have lead to court cases for not being transparent and/or not being adequate.

In this thesis an information system prototype will be constructed to shed light on these processes. The system takes in the structured and unstructered data of the insolvency procedures using open and publicly available data sources[22-28]. From this data the system builds and annotates a complete linked entity structure.

Using a [web application] interface, a stakeholder can query and explore the networks of linked entities and insolvency case data in order to provide transparency on the insolvency processes.
\end{abstract}

%
% The code below should be generated by the tool at
% http://dl.acm.org/ccs.cfm
% Please copy and paste the code instead of the example below.
%
\begin{CCSXML}
<ccs2012>
	<concept>
		<concept_id>10002951.10003317.10003347.10003349</concept_id>
		<concept_desc>Information systems~Document filtering</concept_desc>
		<concept_significance>300</concept_significance>
	</concept>
</ccs2012>
\end{CCSXML}
\ccsdesc[300]{Information systems~Document filtering}

\keywords{keyword1, keyword2, keyword3}

\maketitle

\input{samplebody-conf}

\bibliographystyle{ACM-Reference-Format}
%\bibliography{sample-bibliography}

\end{document}
