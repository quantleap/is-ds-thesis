\section{Introduction}

Bevat je onderzoeksvraag (of vragen)
• Plaatst je vraag in de bestaande literatuur.
Je onderzoeksvraag is leidend voor je hele scriptie. Alles wat je doet moet uiteindelijk terug te voeren zijn op 1 doel: het beantwoorden van die vraag.
Typisch zal je het dan ook zo doen:
Mijn onderzoeksvraag is onderverdeeld in de volgende deelvragen:
RQ1 . . . We beantwoorden deze vraag door het volgende te doen/ antwoord op de volgende vragen te vinden/ . . .
1. Vragen op dit niveau kan je echt beantwoorden, en dat doe je in je Evaluatie sectie 4.
RQ2...
RQ3...
Je Evaluatie sectie 4 bevat evenveel subsecties als je deelvragen hebt. En in elke sectie beantwoord je dan die deelvraag met behulp van de vragen op het onderste niveau.
In je conclusies kan je dan je hoofdvraag gaan beantwoorden op basis van al het eerder vergaarde bewijs.

\section{Related Work}
Deze sectie bestaat uit een aantal ”blokken”, waarin je per blok de relevante literatuur beschrijft.
Neem alleen literatuur op die van belang is voor jouw onderzoeksvraag en deelvragen.
Typisch heb je 1 blok voor je hoofdvraag en per deelvraag RQi een blok.

\subsection{RQ1}

\section{Methodology}
\subsection{Description of the data}
Data verzameling en beschrijving van de data
Hoe is de data verzameld, en hoe heb jij die data verkregen?
Wat staat er in de data? Niet alleen maar een technisch verhaal, maar ook
inhoudelijk. DE lezer moet een goed idee krijgen over de technische inhoud en wat het betekent.

\subsection{Methods}
Hoe je je vraag gaat beantwoorden.
Dit is de langste sectie van je scriptie.
Als iets erg technisch wordt kan je een deel naar de Appendix verplaatsen. Probeer er een lopend verhaal van te maken.
Het is heel handig dit ook weer op te delen nav je deelvragen
\subsubsection{RQ1}

\section{Evaluation}
Met een subsectie voor elke deelvraag.
In hoeverre is je vraag beantwoord?
Een mooie graphic/visualisatie is hier heel gewenst. Hou het kort maar krachtig.

\section{Conclusions}
Hierin beantwoord je jouw hoofdvraag op basis van het eerder vergaarde bewijs.

\subsection{Acknowledgements}
Hier kan je bedanken wie je maar wilt.

References
Slides?

