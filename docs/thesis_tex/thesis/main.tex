\documentclass[sigconf]{acmart}

%\usepackage{booktabs} % For formal tables
%\usepackage[utf8]{inputenc}

% Copyright
\setcopyright{rightsretained}

\title{Armchair Auditing of Insolvency Processes}
\author{Tom Akkermans}
\affiliation{\institution{University of Amsterdam}}
\email{tom.akkermans@student.uva.nl}


\begin{document}
\begin{titlepage}
\begin{center}

\textsc{\Large Armchair Auditing of Insolvency Procedures }

\bigskip

\textsc{\large
submitted in partial fulfilment for the degree of master of science\\
%
\bigskip
Tom Akkermans\\
%
11323671\\
%
\bigskip
master information studies\\
%
data science \\
%
faculty of science\\
%
university of amsterdam\\
%
\bigskip
%Your Date of defence in the format YYYY-MM-DD
2018-08-18
}

\end{center}
 

\vfill

% In case of an internal project, remove External Supervisor or if you had two internal supervisors, change the header into 
%  & First Supervisor & Second Supervisor  \\
\begin{center}
\begin{tabular}{|l||ll|}
\hline
 & \textbf{First Supervisor} & \textbf{Second Supervisor}  \\   
 \hline
\textbf{Title, Name} & Dr Maarten Marx&  \\
\textbf{Affiliation} &UvA, FNWI, IvI & \\ 
\textbf{Email} & maartenmarx@uva.nl& . \\
\hline
\end{tabular}
\end{center}

\bigskip

% logos
\begin{center}
\mbox{\includegraphics[width=.2\paperwidth]{logo-uva.png} 
\includegraphics[width=.2\paperwidth]{ads.png}
}
\end{center}
\end{titlepage}

%
%\newpage
%
%\end{document}


\begin{abstract}
When a company is declared bankrupt at the court, a court committee will appoint a curator to settle the bankruptcy. The curator's task is to redeem the creditors claims as much as possible by monetizing the company's estate. A court's judge supervises the curator to act in the best interest of the creditors.

The curator appointment process, the judge supervision capability and the curator-creditor interest alignment have been topic of research[[1-2]], raised concerns in the media \cite{dennis_meneer_2018:1} and have lead to court cases for not being transparent and/or not being adequate. Involved parties such as judges and curators as well as the creditors and insolvent itself would benefit from more transparency and information on these processes.

The Dutch government has initiated a transparency program for the legal system. It provides on-line search form for a single insolvency case and provides open data sources with technical APIs supplying court publications and curator reports. But the search form is limited and not everyone with an interest can handle the raw data of the APIs. Instead of open data, the stakeholder needs open analysis to enable armchair auditing of insolvency processes to answer their questions.

In this thesis an information system prototype will be developed to enable the auditing of insolvency processes for non tech savvy users. The system takes in structured and unstructured data of insolvency procedures using open and publicly available data sources[[22-28]]. From this data the system builds and annotates a complete linked entity structure and unstructured data is made searchable on multiple levels. The stakeholder can query the linked data with a non-technical interface in order to explore insolvency processes and answer specific questions.
\end{abstract}

%
% The code below should be generated by the tool at
% http://dl.acm.org/ccs.cfm
% Please copy and paste the code instead of the example below.
%
\begin{CCSXML}
<ccs2012>
	<concept>
		<concept_id>10002951.10003317.10003347.10003349</concept_id>
		<concept_desc>Information systems~Document filtering</concept_desc>
		<concept_significance>300</concept_significance>
	</concept>
</ccs2012>
\end{CCSXML}
\ccsdesc[300]{Information systems~Document filtering}

\keywords{keyword1, keyword2, keyword3}

\maketitle 


\section{Introduction}

Bevat je onderzoeksvraag (of vragen)
• Plaatst je vraag in de bestaande literatuur.
Je onderzoeksvraag is leidend voor je hele scriptie. Alles wat je doet moet uiteindelijk terug te voeren zijn op 1 doel: het beantwoorden van die vraag.
Typisch zal je het dan ook zo doen:
Mijn onderzoeksvraag is onderverdeeld in de volgende deelvragen:
RQ1 . . . We beantwoorden deze vraag door het volgende te doen/ antwoord op de volgende vragen te vinden/ . . .
1. Vragen op dit niveau kan je echt beantwoorden, en dat doe je in je Evaluatie sectie 4.
RQ2...
RQ3...
Je Evaluatie sectie 4 bevat evenveel subsecties als je deelvragen hebt. En in elke sectie beantwoord je dan die deelvraag met behulp van de vragen op het onderste niveau.
In je conclusies kan je dan je hoofdvraag gaan beantwoorden op basis van al het eerder vergaarde bewijs.

\section{Related Work}
Deze sectie bestaat uit een aantal ”blokken”, waarin je per blok de relevante literatuur beschrijft.
Neem alleen literatuur op die van belang is voor jouw onderzoeksvraag en deelvragen.
Typisch heb je 1 blok voor je hoofdvraag en per deelvraag RQi een blok.

\subsection{RQ1}

\section{Methodology}
\subsection{Description of the data}
Data verzameling en beschrijving van de data
Hoe is de data verzameld, en hoe heb jij die data verkregen?
Wat staat er in de data? Niet alleen maar een technisch verhaal, maar ook
inhoudelijk. DE lezer moet een goed idee krijgen over de technische inhoud en wat het betekent.

\subsection{Methods}
Hoe je je vraag gaat beantwoorden.
Dit is de langste sectie van je scriptie.
Als iets erg technisch wordt kan je een deel naar de Appendix verplaatsen. Probeer er een lopend verhaal van te maken.
Het is heel handig dit ook weer op te delen nav je deelvragen
\subsubsection{RQ1}

\section{Evaluation}
Met een subsectie voor elke deelvraag.
In hoeverre is je vraag beantwoord?
Een mooie graphic/visualisatie is hier heel gewenst. Hou het kort maar krachtig.

\section{Conclusions}
Hierin beantwoord je jouw hoofdvraag op basis van het eerder vergaarde bewijs.

\subsection{Acknowledgements}
Hier kan je bedanken wie je maar wilt.

References
Slides?



\bibliographystyle{ACM-Reference-Format}
\bibliography{bibliography}

\end{document}
