\begin{abstract}
When a company is declared bankrupt by the court, a court committee appoints a administrator to settle the bankruptcy. The administrator's task is to cash in company's estate and use it to settle the creditors claims. A court's judge must supervise the administrator in order to to act in the best interest of the creditors.

The supervisory function of the judge, the conflict of interest between the administrator and creditors and the appointment process of the administrator are processes which have been the subject of research\cite{boluk_2011}, about which media articles\cite{dennis_meneer_2018:1, dennis_meneer_2017:1, jan-hein_strop_2015:1} have appeared and which have led to legal proceedings. The involved parties ask for more transparency and information about these processes. Access for the general public and independent media to this process information could provide an extra check and thus further improvement of processes.

The Dutch government started a transparency program in 2005 publishing insolvency data\cite{rechtspraak:1}. It provides an on-line search form to retrieve a single insolvency case and provides open data sources with technical APIs that provide court publications and administrator reports. However, the information from a single insolvency case is limited, the report information is unstructured and not all interested parties can deal with the raw data of the APIs. Instead of open data, there is a need for open analysis to enable an 'armchair audit'\cite{o_leary_2015} of insolvency processes to answer their questions.

In this thesis a prototype of an information system will be presented that enables the audit of insolvency processes for non-technical stakeholders. The system processes large amounts of structured and unstructured data of insolvency processes using open and publicly available data sources. From this data, the system builds and annotates a fully linked, clean entity structure of administrator reports, court publications, insolvencies, trustees and judges. Unstructured report data is made searchable at multiple levels and process data is uncovered. 

We show that the system can provide valuable insights to the interested party who can use a non-technical interface to investigate the insolvency processes and answer specific questions. 
\end{abstract}