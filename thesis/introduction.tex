\section{Introduction}
When a company is declared bankrupt by the court, a court committee appoints a administrator to settle the bankruptcy. The administrator's task is to cash in the company's estate and use it to settle the creditors claims. A court's judge must supervise the administrator in order to to act in the best interest of the creditors.

The supervisory function of the judge, the conflict of interest between the administrator and creditors and the appointment process of the administrator are processes which have been the subject of research\cite{boluk_2011}, about which media articles have appeared \cite{dennis_meneer_2018:1, dennis_meneer_2017:1, jan-hein_strop_2015:1} and which have led to legal proceedings. The involved parties ask for more transparency and information about these processes. Supervisory judges working in a reactive mode under the work pressure could benefit from data driven supervision. Access for the general public and journalists to this process information could provide an extra check and thus further improvement of insolvency processes.

The Dutch government started in 2005 publishing insolvency data\cite{rechtspraak:1} according to the insolvency law \cite{law:1}. It provides an on-line search form to retrieve a single insolvency case and provides open data sources with technical APIs that provide court publications and administrator reports. However, the information from a single insolvency case is limited as it does not provide aggregated information, the administrator reports are unstructured and the collection not searchable and not all interested parties can deal with the offered raw data APIs. 

Instead of open data, there is a need for \textbf{open analysis} to enable 'armchair audits'\cite{o_leary_2015} of insolvency processes. In this thesis a prototype of an information system will be presented that enables such audits and search for non=technical users. For this, the system processes large amounts of structured and unstructured data of insolvency processes using open and publicly available data sources. From this data, the system:
\begin{itemize}
\item extracts insolvency process flow information.
\item builds a fully linked, clean entity structure of insolvents, administrators, judges, courts as well as administrator reports and court publications.
\item extracts the text of administrator reports and indexes sections and parameters by imposing structure on the content.
\end{itemize} 

A web GUI on top of the data models provides the user interface for audit and search. We describe the implementation challenges and show that the system can provide new insights to the stakeholders who can use a non-technical interface to investigate the insolvency processes and answer specific questions. 

\subsection{Parties involved in the insolvency process}
There are many parties involved in the insolvency process. Directly involved are the the administrator, the insolvent, the judge and the creditors. Indirectly involved but interested are a.o. the organisations of Recofa and Insolad and journalists. 

\subsubsection{The Administrator (\textit{De Curator})}
The administrator's task is to liquidate the bankrupt firm by selling the assets and from the proceeds pay off the creditors. A second task is to investigate the default and see if there is a case of mismanagement or fraud.

\subsubsection{The Insolvent} (\textit{De Failliet})
The insolvent is declared bankrupt by a creditor or declared himself bankrupt. He might be interested in continuing the business and would not want to be persecuted for fraud or malpractice.

\subsubsection{The Bankruptcy Judge (\textit{De Rechter-Commissaris})}
The supervisory judge exercises supervision over the administrator which is appointed by the court and is entitled to grant him or her permissions for certain actions.

\subsubsection{The Creditors}( \textit{De Schuldeisers})
The creditor has a claim on the insolvent. There is a ranking of creditors from preferred creditors to unsecured creditors.

\subsubsection{Insolad}
Insolad is the association of lawyers in insolvency law (administrators). It provides up-to-date knowledge on the insolvency practise and the development of laws and regulations in the field.

\subsubsection{\textit{Rechters-Commissarissen Insolventies} (Recofa)}
Recofa is the consultative body under the council of justice, \textit{raad van de rechtspraak}, specifically for judges working in insolvency law. It sets out policy guidelines for the courts and instructions (e.g. on reporting) for the administrators. The council of the justice system is tasked with improving the quality and efficacy of the jurisdiction. It performs research and shares research findings. It also supplies IT resources.

\subsubsection{The Investigative Journalist}
The investigative journalist is interested in informing the public on the balance of powers with the insolvency law and typically writes critically about cases where this balance is disrupted.

\subsubsection{The Economic Journalist}
The economic journalist is interested on the impact of the insolvencies and trends therein on the economy .
