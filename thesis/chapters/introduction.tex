\section{Introduction}
When a company is declared bankrupt by the court, a court committee appoints an administrator to settle the bankruptcy. The administrator's task is to liquidate the company's estate and use the proceeds to settle the creditors claims. A supervisory judge ensures that the administrator is acting in the best interest of the creditors. 

The supervisory function of the judge, the conflict of interest between the administrator and creditors and the appointment process of the administrator are processes which have been the subject of research\cite{boluk_2011}, about which media articles have appeared \cite{dennis_meneer_2018:1, dennis_meneer_2017:1, jan-hein_strop_2015:1} and which have led to legal proceedings \todo[inline]{legal proceedings citation here}. \todo[inline]{describe additional points of interest}

The involved parties demand more transparency of these processes. Supervisory judges working in a reactive mode under the work pressure could benefit from data driven supervision. Information access to the general public and journalists to this processes  would provide additional checks and balances to further process improvement.

The Dutch government started in 2005 publishing insolvency data\cite{rechtspraak:1} according to the insolvency law \cite{law:1}. It provides an on-line search form \cite{rechtspraak:4} to retrieve a single insolvency case and provides open data web services to provide court publications and administrator reports in XML and PDF format. However, the information from a single insolvency case is limited as it does not provide aggregated and linked information. The administrator reports are unstructured and not searchable and not all interested parties can deal with the offered raw data APIs. 

Instead of open data, there is a need for open analysis to enable 'armchair audits'\cite{o_leary_2015} of insolvency processes. In this thesis we investigate \textbf{whether it is possible to build a complete and correct structured information system based on open and public data that is useful in that it enables such audits and search for non-technical users.}

We describe the steps in building such a system that takes in large amounts of open and publicly available data in structured and unstructured data form, extracts and enriches useful facts and makes it consumable for analysis to provide insights into the insolvency processes.
