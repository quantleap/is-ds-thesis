\subsection{Constructing the Entity Network}
The requested CIR XML data will go through an Extract Transform Load (ETL) process. Here each entity data is extracted, de-duplicated and identified and linked to the other entities to construct the entity network. For each entity a so-called master data table is created that contains the unique and agreed upon instances or 'golden records' of the entities. 

\subsubsection{Insolvency Cases}
Insolvency cases are supplied with a natural identifier (\textit{insolventienummer}), for example F.19/13/123, which consist of the insolvency type (F), a system number (19), year of insolvency ([20]13) and a serial number (123). 

Duplicate insolvency cases exist for a number of reasons:
\begin{enumerate}
	\item data errors: padded zeros on the serial number, e.g. '0123'
	\item redefinition of courts\footnote{in 2013 courts were consolidated (\textit{Herziening Gerechtelijke Kaart}, see \cite{om:1}).}: cases administered both with the new case number format as well as the old case number format without the system number.
	\item cases handed over to another court.
\end{enumerate}
To de-duplicate these cases, lineage of erred or transferred cases was established and publications and reports re-linked to the correct case. The duplicate cases could then easily be filtered out.

\subsubsection{Judges}
We define the master data of judges by scraping the names of judges found in the register of ancillary positions of judges\cite{rechtspraak:nevenfuncties}. The register names and CIR judge names are normalized after which the CIR name is linked to the most probable register judge name using the Levenshtein distance \todo[inline]{describe what happens in case of a tie}.

\paragraph{CIR judge name normalization and de-duplication}
Judge names in the CIR data are provided as single free text field and can contain many duplicates. For example one judge's name appears as:

\begin{itemize}
\item "mr. W.J. Geurts - de Veld"
\item "mr. W.J. Geurts-deVeld"
\item "mr. W.J. Geurts-de Veld"
\item "mr. W.J.Geurts-de Veld"
\item "mr.W.J. Geurts-de Veld"
\item "mr. W.J. Geurts-de Veld (Rotterdam)"
\item "mr W.J.Geurts-de Veld"
\item "W.J.Geurts-de Veld"
\end{itemize}

This data needs to be de-duplicated and we do this by cleaning and normalizing the name in a number of sequential steps where the order is of importance:
\begin{enumerate}
\item add missing spaces between name parts
\item change the name to lower case and strip surrounding white-space characters
\item remove accents, academic and nobility titles, additions in brackets, periods and double spaces
\item normalize the use of hyphens
\end{enumerate}

\paragraph{register judge normalization}
The normalization of register judge names slight differs from the CIR name. The number of removed nobility and academic titles is larger and ...

\todo[inline]{simple matching using Levenshtein distance does not place the same weight on initials and surname, describe matching}


\paragraph{Administrators}

\subsection{Measuring CIR Data Completeness}
To measure the completeness of the data obtained from the CIR XML web service we make use of its following methods\cite{rechtspraak:tech-docs}:
\begin{enumerate}
\item \textbf{searchModifiedSince} (cutoff 2012-01-01 / 2005-01-01) : returns the publication ids of modified insolvents
\item \textbf{searchReportsSince} (cutfoff 2010-07-01? / 2005-01-01) : returns the report ids of added reports.
\end{enumerate}

The publications contain the fields for the ids or names of the other entities. By checking these fields form missing values and the values against the database we measure completeness of the data.
\todo[inline]{(re)measure completeness using dates above}

\subsection{Extracting Information from Entity Data}
\todo[inline]{write}
\subsection{Extracting Information from Unstructured Reports}
\todo[inline]{write}


\begin{comment}
\paragraph{Database normalization}
This leads to unwanted duplication as names can be written in many different forms and typos can be introduced, 

We need to define so-called master data for judges and administrators containing the real world entities. The two data sources in the sections \ref{NOvA Tableau} and \ref{Nevenfuncties Rechters} are chosen for this purpose. CIR entity names are first normalized for de-duplication and are subsequently linked to the master data records on their normalized name. 

\paragraph{PDF report web service}




The number of current defaults is about the lowest of the century [ref] [graph of declining defaults]






\subsection{Implementation challenges}
\subsubsection{Entity De-duplication, Scraping and Linking}
\paragraph{Deduplication: normalizing names}
CIR provides administrator names in four parts: title, initials, middle part and family name. 

\textbf{The initials} were normalized into single characters using periods and no spaces. In Dutch, initials for names like \textit{Theodoor} and \textit{Christiaan} are written with two or three characters as \textit{Th.} or \textit{Chr.}. These names are derived from the Greek where the initial \textit{Th(eta)} for example is written as one character. Initials for double names like \textit{Albert-Jan} may be written as \textit{A-J.} or \textit{A.-J.}. 

\textbf{The middle part}, e.g. 'van der' or 'ter' was sometimes supplied in the middle part field and sometimes in the family name (and sometimes in both). To normalize the middle part it was merged into the family name making sure it did not appear twice.

\textbf{The family name} was stripped from academic and noble (not uncommon in this profession) titles. Misplaced initials and CIR specific comments were removed. Maiden names are connected to the surname with a hyphen and no spaces.

Furthermore, all parts are made lower case, surrounding whitespace characters are stripped, accents are removed and multiple spaces are replaces with a single space. 

De-duplication by normalizing the administrator names reduced the number of distinct administrator names from 2329 to 1559, a 33\% reduction.
\\\\
Judge names are provided in a single field. This sometimes leads to initials or a middle name part that is stuck together to the family name and was separated using regexm e.g. in 'C. vanSteenderen' and 'mr. W.J.Geurts-de Veld'. CIR often adds the court between parenthesis which was also removed in the normalization.

De-duplication by normalizing the judge names reduced the number of distinct judge names from 565 to 277, a 51\% reduction.

\paragraph{Administrator name scraping}
NOvA unfortunately does not provide complete lists of registered lawyers to which we could link the administrators. Instead we link the administrator by using the site's search form. An underlying REST API returning JSON data was discovered which was used instead of the form. The API (or form) can not handle initials or stop words as 'de' and 'van' and only the family name without stop words was used. The best candidate in the search results, which are sorted on the relevance provided by NOvA, is found by (partially) matching the initials until a match is found. The API enables the use of filters and we filter from narrow to broad until a candidate is found, filtering on: 
\begin{enumerate}
\item{legal specialism: administrator}
\item{legal branch: insolvency law}
\item{no filter}
\end{enumerate}

NOvA only supplies the actual lawyer register but CIR also lists administrators previously working on a case therefore a substantial amount of administrators cannot be linked. Another website, www.advocatenzoeken.nl that sources data from NOvA but keeps historical data was scraped as well to complement NOvA.

\paragraph{Administrator name linking}
From the normalized name list we linked 73\% to NOVa registered lawyers and another 18\% administrators from the www.advocatenzoeken.nl site. The remaining 10\% unfound names are linked to a special 'unknown' lawyer.

\begin{table}[h]
\caption{Administator entity linking results}
\centering
\begin{tabular}{l r r r}
\hline\hline
Source & no. linked & correct & incorrect\\
\hline
NOvA & 1134 & 1133 & 1\\
AdvocatenZoeken & 280 & 279 & 1 \\
Not found & 149 &&\\
\hline
Total & 1563 & 1412 &\\
\hline
\end{tabular}
\label{table:administrator_linking}
\end{table}

Correctness is assumed when the names match. When an obvious typo is found in the family name. When the initials either match, or are contained in the master record or are permutated or are missing. 

[discuss not found results: common name, place in name]

\paragraph{Judge name scraping}
The CIR register for ancillary functions for judges was scraped for judges from courts and higher courts. The total set contains 3691 judges and should be a superset of active case judges. The website was driven by the Angular javascript framework and could not easily be scraped with simple HTTP requests. Selenium was used to mimic user browsing behaviour to invoke the javascript. 

\paragraph{Judge entity linking}
A judge is linked by to a judge master record with the most similar normalized name where similarity was calculated using the Levensthein or edit distance. 88\% of the normalized names were matched. 4 of the incorrectly 12\% matched names could be set manually and the remaining 8\% was linked to a special 'unknown' judge.


\begin{table}[h]
\caption{Judge entity linking results}
\centering
\begin{tabular}{l r r r}
\hline\hline
Source & no. linked & correct & incorrect\\
\hline
Ancillary positions register & 277 & 246 & 32\\
+Manual lookup of the 32 &   & 10 & -10\\
\hline
Total & 277 & 256 & 22\\
\hline
\end{tabular}
\label{table:judge_linking}
\end{table}

Correctness is defined similar to the administrator normalized name correctness.

[discuss not found (initially) results: maiden name, not a judge anymore]

\subsubsection{Process Mining}

\subsubsection{PDF content extraction}
The PDF reports can be split into two types depending on how the PDF was created:
\begin{enumerate}
\item Scanned from paper using a scanner or copier.
\item Converted by software from another format.
\end{enumerate}

\paragraph{Scanned PDFs}
Scanned PDFs contain images only. To convert the images to text we used the Ghostscript library for image extraction and the Tesseract OCR engine for the character recognition. Tesseract supports the Dutch language which is paramount to our application. It is very important to supply Tesseract with good quality images. We used the Ghostscript settings of a \emph{tiffgray} output device with a \emph{300 DPI} resolution. 

\paragraph{Converted PDFs}
Converted PDF content can be extracted with a number of packages such as PDFMiner, PyPDF2, pyPoppler and pdftotext. We chose the latter after comparison because it maintains the layout which is important for section and parameter extraction and being build in C++ it is very fast.

\paragraph{Third type}
A third type appeared: PDFs that were scanned and subsequently OCR-ed by a copier. They contain both text and images. The OCR quality is often poor. We retried re-OCR-ing the images with Tesseract which solved some errors but introduced others. Meta data about a.o. the scanner type was obtained which could improve post-process text extraction in future work.


\subsubsection{PDF report section and parameter extraction}
[todo]

\end{comment}



