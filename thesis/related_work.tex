\section{Related Work}
[todo]

\subsection{Research Questions}

The main research question is:
\textbf{Is is possible to build a useful, complete and correct structured information system based on unstructured CIR data?}

The information system is useful when it answers questions of the parties involved. We define so-called Persona to represent archetypical users of the information system and define specific questions they have. These questions are distilled from the insolvency law, news articles, research papers, court cases [refs] and interviews.

\subsubsection{Persona}
The main Persona are: 
\begin{itemize}
\item \textbf{The Judge} representing the side of Recofa, \textit{Raad van de Rechtspraak} and The Bankruptcy Judge (\textit{Rechter-Commissaris})
\item \textbf{The Administrator} representing the side of Insolad (Vereniging Insolventierecht Advocaten) and The Administrator (Curator) 
\item \textbf{The Insolvent} representing the owner(s) of the defaulted company.
\item \textbf{The Creditor} representing the Unsecured Creditor (\textit{De Concurrente Schuldeiser}).
\item \textbf{The Journalist} representing the investigative and economic journalist.
\end{itemize}

In the following sections each Persona will be described and their questions to the system stated.

\subsubsection{The Judge}
The Persona of Judge is interested in the adherence to the insolvency law and to the Recofa policy guidelines. It is also interested in specific trends in the insolvency processes and operational issues such as work pressure for judges. On an individual judge level it is interested in supervising its active cases.

Questions:
\begin{itemize}
	\item How many cases does a judge supervise at a certain point in time (now)
	\begin{itemize}
		\item … distribution over all judges at a certain court
		\item … distribution over all judges at all courts
	\end{itemize}
	\item What is the process flow of cases through court: 
	\begin{itemize}
		\item What is the case time distribution from begin to end
		\item How long does it take to set the verification meeting (law says < 14 days)
		\item How long does it take to publish the plan of final distribution.
		\item What percentage of cases:
		\begin{itemize}
			\item end early as there are no assets
			\item end by paying all creditors
			\item contain an agreement between insolvent and creditors (akkoord)
			\item are following an simplified settlement (no meeting of creditors)	
			\item are filed by the insolvent vs creditor
		\end{itemize}
		\item What are the experience factor (\textit{jaren praktijkervaring}) and estate factor (\textit{grootte actief}) for the cases determining the administrators hourly wage.
	\end{itemize}
	
	\item Are administrators reporting according to the instructions:
	\begin{itemize}
		\item How often is the progress reporting deadline breached.
		\item How often is the financial attachment omitted for all reports in a case.
	\end{itemize}
	\item Are administrators using the supplied template for progress report:
	\item How often does insolvency fraud occur
	\item \textit{What are the issues in progress reports that need the judge's attention}
\end{itemize}

\subsubsection{The Administrator}
The Persona of Administrator is interested in the administrator appointment process of the courts and on issues that threaten its business or leave it powerless.\\

from \cite{samr_2017:1}:\\
\textit{"Volgens de curatoren is meer transparantie van belang
bij de verdeling van faillissementen: ze ervaren deze
verdeling nu als een black box. [...] Daarnaast maken de curatoren
zich flinke zorgen over het hoge verloop onder rechtercommissarissen
en de griffie en de gevolgen daarvan."}\\

Questions:
\begin{itemize}
	\item Which curators are on the court’s short list for appointment
	\item Are insolvency cases distributed fairly over the curators on the list
	\item How long are judges working in the insolvency field, how often do they rotate.
	\item How often do empty estates occur (an empty estate implicated work without pay)
	\item To what degree do Banks claim all the proceeds from the assets.
\end{itemize}

\subsubsection{The Insolvent}
The Persona of Insolvent could be interested in restarting the business. It also is afraid of being made personally liable for the bankruptcy claims.

\begin{itemize}
	\item -	How often are insolvent made personally liable (when there is little estate)
	\item -	How often do insolvency restarts and prepacks occur
\end{itemize}

\subsubsection{The Creditor}
The Persona of Creditor is interested in the recovery rate of its claim and the time of payout and the factors that influence those.

\begin{itemize}
	\item What is my expected recovery rate and time
	\item -	How high is the administrator’s salary, is he eating up the proceeds.
\end{itemize}

\subsubsection{The Journalist}
The Persona of Journalist is writing an article on a certain insolvency phenomenon. An economic journalist might periodically write the same story on say the number of defaults in the last quarter compares to a previous period, usually for business readers. The investigative journalist writes for a wide audience, therefore the article contains both general trends as well as the personal individual story. For this the investigative journalist needs query functionality to dig for evidence and pull individual records.

\begin{itemize}
	\item J2: (topic: incapable administrators) Which administrators have often been taken of their cases by the judge.
	\item J1: How many defaults occurred over the last three months compared to a year earlier. [viz:cumulative lines of two periods]
\end{itemize}
